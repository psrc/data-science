% Options for packages loaded elsewhere
\PassOptionsToPackage{unicode}{hyperref}
\PassOptionsToPackage{hyphens}{url}
%
\documentclass[
]{article}
\usepackage{amsmath,amssymb}
\usepackage{lmodern}
\usepackage{ifxetex,ifluatex}
\ifnum 0\ifxetex 1\fi\ifluatex 1\fi=0 % if pdftex
  \usepackage[T1]{fontenc}
  \usepackage[utf8]{inputenc}
  \usepackage{textcomp} % provide euro and other symbols
\else % if luatex or xetex
  \usepackage{unicode-math}
  \defaultfontfeatures{Scale=MatchLowercase}
  \defaultfontfeatures[\rmfamily]{Ligatures=TeX,Scale=1}
\fi
% Use upquote if available, for straight quotes in verbatim environments
\IfFileExists{upquote.sty}{\usepackage{upquote}}{}
\IfFileExists{microtype.sty}{% use microtype if available
  \usepackage[]{microtype}
  \UseMicrotypeSet[protrusion]{basicmath} % disable protrusion for tt fonts
}{}
\makeatletter
\@ifundefined{KOMAClassName}{% if non-KOMA class
  \IfFileExists{parskip.sty}{%
    \usepackage{parskip}
  }{% else
    \setlength{\parindent}{0pt}
    \setlength{\parskip}{6pt plus 2pt minus 1pt}}
}{% if KOMA class
  \KOMAoptions{parskip=half}}
\makeatother
\usepackage{xcolor}
\IfFileExists{xurl.sty}{\usepackage{xurl}}{} % add URL line breaks if available
\IfFileExists{bookmark.sty}{\usepackage{bookmark}}{\usepackage{hyperref}}
\hypersetup{
  pdftitle={dissim\_sf\_share},
  pdfauthor={suzanne},
  hidelinks,
  pdfcreator={LaTeX via pandoc}}
\urlstyle{same} % disable monospaced font for URLs
\usepackage[margin=1in]{geometry}
\usepackage{color}
\usepackage{fancyvrb}
\newcommand{\VerbBar}{|}
\newcommand{\VERB}{\Verb[commandchars=\\\{\}]}
\DefineVerbatimEnvironment{Highlighting}{Verbatim}{commandchars=\\\{\}}
% Add ',fontsize=\small' for more characters per line
\usepackage{framed}
\definecolor{shadecolor}{RGB}{248,248,248}
\newenvironment{Shaded}{\begin{snugshade}}{\end{snugshade}}
\newcommand{\AlertTok}[1]{\textcolor[rgb]{0.94,0.16,0.16}{#1}}
\newcommand{\AnnotationTok}[1]{\textcolor[rgb]{0.56,0.35,0.01}{\textbf{\textit{#1}}}}
\newcommand{\AttributeTok}[1]{\textcolor[rgb]{0.77,0.63,0.00}{#1}}
\newcommand{\BaseNTok}[1]{\textcolor[rgb]{0.00,0.00,0.81}{#1}}
\newcommand{\BuiltInTok}[1]{#1}
\newcommand{\CharTok}[1]{\textcolor[rgb]{0.31,0.60,0.02}{#1}}
\newcommand{\CommentTok}[1]{\textcolor[rgb]{0.56,0.35,0.01}{\textit{#1}}}
\newcommand{\CommentVarTok}[1]{\textcolor[rgb]{0.56,0.35,0.01}{\textbf{\textit{#1}}}}
\newcommand{\ConstantTok}[1]{\textcolor[rgb]{0.00,0.00,0.00}{#1}}
\newcommand{\ControlFlowTok}[1]{\textcolor[rgb]{0.13,0.29,0.53}{\textbf{#1}}}
\newcommand{\DataTypeTok}[1]{\textcolor[rgb]{0.13,0.29,0.53}{#1}}
\newcommand{\DecValTok}[1]{\textcolor[rgb]{0.00,0.00,0.81}{#1}}
\newcommand{\DocumentationTok}[1]{\textcolor[rgb]{0.56,0.35,0.01}{\textbf{\textit{#1}}}}
\newcommand{\ErrorTok}[1]{\textcolor[rgb]{0.64,0.00,0.00}{\textbf{#1}}}
\newcommand{\ExtensionTok}[1]{#1}
\newcommand{\FloatTok}[1]{\textcolor[rgb]{0.00,0.00,0.81}{#1}}
\newcommand{\FunctionTok}[1]{\textcolor[rgb]{0.00,0.00,0.00}{#1}}
\newcommand{\ImportTok}[1]{#1}
\newcommand{\InformationTok}[1]{\textcolor[rgb]{0.56,0.35,0.01}{\textbf{\textit{#1}}}}
\newcommand{\KeywordTok}[1]{\textcolor[rgb]{0.13,0.29,0.53}{\textbf{#1}}}
\newcommand{\NormalTok}[1]{#1}
\newcommand{\OperatorTok}[1]{\textcolor[rgb]{0.81,0.36,0.00}{\textbf{#1}}}
\newcommand{\OtherTok}[1]{\textcolor[rgb]{0.56,0.35,0.01}{#1}}
\newcommand{\PreprocessorTok}[1]{\textcolor[rgb]{0.56,0.35,0.01}{\textit{#1}}}
\newcommand{\RegionMarkerTok}[1]{#1}
\newcommand{\SpecialCharTok}[1]{\textcolor[rgb]{0.00,0.00,0.00}{#1}}
\newcommand{\SpecialStringTok}[1]{\textcolor[rgb]{0.31,0.60,0.02}{#1}}
\newcommand{\StringTok}[1]{\textcolor[rgb]{0.31,0.60,0.02}{#1}}
\newcommand{\VariableTok}[1]{\textcolor[rgb]{0.00,0.00,0.00}{#1}}
\newcommand{\VerbatimStringTok}[1]{\textcolor[rgb]{0.31,0.60,0.02}{#1}}
\newcommand{\WarningTok}[1]{\textcolor[rgb]{0.56,0.35,0.01}{\textbf{\textit{#1}}}}
\usepackage{graphicx}
\makeatletter
\def\maxwidth{\ifdim\Gin@nat@width>\linewidth\linewidth\else\Gin@nat@width\fi}
\def\maxheight{\ifdim\Gin@nat@height>\textheight\textheight\else\Gin@nat@height\fi}
\makeatother
% Scale images if necessary, so that they will not overflow the page
% margins by default, and it is still possible to overwrite the defaults
% using explicit options in \includegraphics[width, height, ...]{}
\setkeys{Gin}{width=\maxwidth,height=\maxheight,keepaspectratio}
% Set default figure placement to htbp
\makeatletter
\def\fps@figure{htbp}
\makeatother
\setlength{\emergencystretch}{3em} % prevent overfull lines
\providecommand{\tightlist}{%
  \setlength{\itemsep}{0pt}\setlength{\parskip}{0pt}}
\setcounter{secnumdepth}{-\maxdimen} % remove section numbering
\ifluatex
  \usepackage{selnolig}  % disable illegal ligatures
\fi

\title{dissim\_sf\_share}
\author{suzanne}
\date{9/21/2021}

\begin{document}
\maketitle

\hypertarget{dissimilarity-indices-vs-share-of-single-family-zoning}{%
\subsection{Dissimilarity indices vs share of single family
zoning}\label{dissimilarity-indices-vs-share-of-single-family-zoning}}

In this analysis, we are comparing the dissimilarity indices at the
block group level to zoning. Because the zoning and the block groups
don't match up, we found the residentail share of the block group
defined as single family (\textgreater8.7 DU per acre)

\begin{Shaded}
\begin{Highlighting}[]
\FunctionTok{setwd}\NormalTok{(}\StringTok{"\textasciitilde{}/GitHub/data{-}science/dissimilarity"}\NormalTok{)}

\FunctionTok{library}\NormalTok{(broom)}
\end{Highlighting}
\end{Shaded}

\begin{verbatim}
## Warning: package 'broom' was built under R version 4.0.5
\end{verbatim}

\begin{Shaded}
\begin{Highlighting}[]
\FunctionTok{library}\NormalTok{(dplyr)}
\end{Highlighting}
\end{Shaded}

\begin{verbatim}
## Warning: package 'dplyr' was built under R version 4.0.5
\end{verbatim}

\begin{verbatim}
## 
## Attaching package: 'dplyr'
\end{verbatim}

\begin{verbatim}
## The following objects are masked from 'package:stats':
## 
##     filter, lag
\end{verbatim}

\begin{verbatim}
## The following objects are masked from 'package:base':
## 
##     intersect, setdiff, setequal, union
\end{verbatim}

\begin{Shaded}
\begin{Highlighting}[]
\FunctionTok{library}\NormalTok{(corrr)}
\end{Highlighting}
\end{Shaded}

\begin{verbatim}
## Warning: package 'corrr' was built under R version 4.0.5
\end{verbatim}

\begin{Shaded}
\begin{Highlighting}[]
\FunctionTok{library}\NormalTok{(ggplot2)}
\end{Highlighting}
\end{Shaded}

\begin{verbatim}
## Warning: package 'ggplot2' was built under R version 4.0.5
\end{verbatim}

\begin{Shaded}
\begin{Highlighting}[]
\FunctionTok{library}\NormalTok{(tidyverse)}
\end{Highlighting}
\end{Shaded}

\begin{verbatim}
## Warning: package 'tidyverse' was built under R version 4.0.5
\end{verbatim}

\begin{verbatim}
## -- Attaching packages --------------------------------------- tidyverse 1.3.1 --
\end{verbatim}

\begin{verbatim}
## v tibble  3.1.4     v purrr   0.3.4
## v tidyr   1.1.3     v stringr 1.4.0
## v readr   2.0.1     v forcats 0.5.1
\end{verbatim}

\begin{verbatim}
## Warning: package 'tibble' was built under R version 4.0.5
\end{verbatim}

\begin{verbatim}
## Warning: package 'tidyr' was built under R version 4.0.5
\end{verbatim}

\begin{verbatim}
## -- Conflicts ------------------------------------------ tidyverse_conflicts() --
## x dplyr::filter() masks stats::filter()
## x dplyr::lag()    masks stats::lag()
\end{verbatim}

\begin{Shaded}
\begin{Highlighting}[]
\FunctionTok{library}\NormalTok{(Hmisc)}
\end{Highlighting}
\end{Shaded}

\begin{verbatim}
## Warning: package 'Hmisc' was built under R version 4.0.5
\end{verbatim}

\begin{verbatim}
## Loading required package: lattice
\end{verbatim}

\begin{verbatim}
## Loading required package: survival
\end{verbatim}

\begin{verbatim}
## Loading required package: Formula
\end{verbatim}

\begin{verbatim}
## 
## Attaching package: 'Hmisc'
\end{verbatim}

\begin{verbatim}
## The following objects are masked from 'package:dplyr':
## 
##     src, summarize
\end{verbatim}

\begin{verbatim}
## The following objects are masked from 'package:base':
## 
##     format.pval, units
\end{verbatim}

\begin{Shaded}
\begin{Highlighting}[]
\FunctionTok{library}\NormalTok{(DBI)}
\FunctionTok{library}\NormalTok{(odbc)}
\NormalTok{res\_diss}\OtherTok{\textless{}{-}}\FunctionTok{read.csv}\NormalTok{(}\StringTok{\textquotesingle{}res\_diss.csv\textquotesingle{}}\NormalTok{)}
\end{Highlighting}
\end{Shaded}

\begin{Shaded}
\begin{Highlighting}[]
\NormalTok{xvar }\OtherTok{\textless{}{-}} \StringTok{"prct\_sf"}
\NormalTok{dissim\_names}\OtherTok{\textless{}{-}}\FunctionTok{c}\NormalTok{(}\StringTok{\textquotesingle{}White\_\_Black\textquotesingle{}}\NormalTok{, }\StringTok{\textquotesingle{}White\_AIAN\textquotesingle{}}\NormalTok{, }\StringTok{\textquotesingle{}White\_\_API\textquotesingle{}}\NormalTok{,  }\StringTok{\textquotesingle{}White\_\_Hispanic\textquotesingle{}}\NormalTok{, }\StringTok{\textquotesingle{}White\_\_Minority\textquotesingle{}}\NormalTok{,  }\StringTok{\textquotesingle{}Black\_\_AIAN\textquotesingle{}}\NormalTok{,  }\StringTok{\textquotesingle{}Black\_\_API\textquotesingle{}}\NormalTok{,   }\StringTok{\textquotesingle{}Black\_\_Other\_2\_\textquotesingle{}}\NormalTok{,}\StringTok{\textquotesingle{}Black\_\_Hispanic\textquotesingle{}}\NormalTok{,    }\StringTok{\textquotesingle{}AIAN\_\_Asian\textquotesingle{}}\NormalTok{,  }\StringTok{\textquotesingle{}AIAN\_\_Other\_2\_\textquotesingle{}}\NormalTok{,}\StringTok{\textquotesingle{}AIAN\_\_Hispanic\textquotesingle{}}\NormalTok{,  }\StringTok{\textquotesingle{}API\_\_Other\_2\_\textquotesingle{}}\NormalTok{ ,   }\StringTok{\textquotesingle{}API\_\_Hispanic\textquotesingle{}}\NormalTok{,    }\StringTok{\textquotesingle{}Other\_2\_\_\_Hispanic\textquotesingle{}}
\NormalTok{)}
\NormalTok{yvar }\OtherTok{\textless{}{-}} \FunctionTok{names}\NormalTok{(res\_diss)[}\FunctionTok{names}\NormalTok{(res\_diss) }\SpecialCharTok{\%in\%}\NormalTok{ dissim\_names]}

\NormalTok{sig\_test}\OtherTok{\textless{}{-}}\FunctionTok{lapply}\NormalTok{(yvar,}
       \ControlFlowTok{function}\NormalTok{(yvar, xvar, res\_diss)}
\NormalTok{       \{}
         \FunctionTok{print}\NormalTok{(yvar)}
         \FunctionTok{cor.test}\NormalTok{(res\_diss[[xvar]], res\_diss[[yvar]], }\AttributeTok{method=}\StringTok{\textquotesingle{}spearman\textquotesingle{}}\NormalTok{) }\SpecialCharTok{\%\textgreater{}\%}
           \FunctionTok{tidy}\NormalTok{()}
\NormalTok{       \},}
\NormalTok{       xvar,}
\NormalTok{       res\_diss) }\SpecialCharTok{\%\textgreater{}\%}
  \FunctionTok{bind\_rows}\NormalTok{() }\SpecialCharTok{\%\textgreater{}\%}\FunctionTok{mutate}\NormalTok{(}\StringTok{\textquotesingle{}dissimiliarity\_index\textquotesingle{}}\OtherTok{=}\NormalTok{dissim\_names)}
\end{Highlighting}
\end{Shaded}

\begin{verbatim}
## [1] "White__Black"
\end{verbatim}

\begin{verbatim}
## Warning in cor.test.default(res_diss[[xvar]], res_diss[[yvar]], method =
## "spearman"): Cannot compute exact p-value with ties
\end{verbatim}

\begin{verbatim}
## [1] "White_AIAN"
\end{verbatim}

\begin{verbatim}
## Warning in cor.test.default(res_diss[[xvar]], res_diss[[yvar]], method =
## "spearman"): Cannot compute exact p-value with ties
\end{verbatim}

\begin{verbatim}
## [1] "White__API"
\end{verbatim}

\begin{verbatim}
## Warning in cor.test.default(res_diss[[xvar]], res_diss[[yvar]], method =
## "spearman"): Cannot compute exact p-value with ties
\end{verbatim}

\begin{verbatim}
## [1] "White__Hispanic"
\end{verbatim}

\begin{verbatim}
## Warning in cor.test.default(res_diss[[xvar]], res_diss[[yvar]], method =
## "spearman"): Cannot compute exact p-value with ties
\end{verbatim}

\begin{verbatim}
## [1] "White__Minority"
\end{verbatim}

\begin{verbatim}
## Warning in cor.test.default(res_diss[[xvar]], res_diss[[yvar]], method =
## "spearman"): Cannot compute exact p-value with ties
\end{verbatim}

\begin{verbatim}
## [1] "Black__AIAN"
\end{verbatim}

\begin{verbatim}
## Warning in cor.test.default(res_diss[[xvar]], res_diss[[yvar]], method =
## "spearman"): Cannot compute exact p-value with ties
\end{verbatim}

\begin{verbatim}
## [1] "Black__API"
\end{verbatim}

\begin{verbatim}
## Warning in cor.test.default(res_diss[[xvar]], res_diss[[yvar]], method =
## "spearman"): Cannot compute exact p-value with ties
\end{verbatim}

\begin{verbatim}
## [1] "Black__Other_2_"
\end{verbatim}

\begin{verbatim}
## Warning in cor.test.default(res_diss[[xvar]], res_diss[[yvar]], method =
## "spearman"): Cannot compute exact p-value with ties
\end{verbatim}

\begin{verbatim}
## [1] "Black__Hispanic"
\end{verbatim}

\begin{verbatim}
## Warning in cor.test.default(res_diss[[xvar]], res_diss[[yvar]], method =
## "spearman"): Cannot compute exact p-value with ties
\end{verbatim}

\begin{verbatim}
## [1] "AIAN__Asian"
\end{verbatim}

\begin{verbatim}
## Warning in cor.test.default(res_diss[[xvar]], res_diss[[yvar]], method =
## "spearman"): Cannot compute exact p-value with ties
\end{verbatim}

\begin{verbatim}
## [1] "AIAN__Other_2_"
\end{verbatim}

\begin{verbatim}
## Warning in cor.test.default(res_diss[[xvar]], res_diss[[yvar]], method =
## "spearman"): Cannot compute exact p-value with ties
\end{verbatim}

\begin{verbatim}
## [1] "AIAN__Hispanic"
\end{verbatim}

\begin{verbatim}
## Warning in cor.test.default(res_diss[[xvar]], res_diss[[yvar]], method =
## "spearman"): Cannot compute exact p-value with ties
\end{verbatim}

\begin{verbatim}
## [1] "API__Other_2_"
\end{verbatim}

\begin{verbatim}
## Warning in cor.test.default(res_diss[[xvar]], res_diss[[yvar]], method =
## "spearman"): Cannot compute exact p-value with ties
\end{verbatim}

\begin{verbatim}
## [1] "API__Hispanic"
\end{verbatim}

\begin{verbatim}
## Warning in cor.test.default(res_diss[[xvar]], res_diss[[yvar]], method =
## "spearman"): Cannot compute exact p-value with ties
\end{verbatim}

\begin{verbatim}
## [1] "Other_2___Hispanic"
\end{verbatim}

\begin{verbatim}
## Warning in cor.test.default(res_diss[[xvar]], res_diss[[yvar]], method =
## "spearman"): Cannot compute exact p-value with ties
\end{verbatim}

\begin{Shaded}
\begin{Highlighting}[]
\FunctionTok{write.table}\NormalTok{(sig\_test, }\StringTok{"clipboard"}\NormalTok{, }\AttributeTok{sep=}\StringTok{\textquotesingle{}}\SpecialCharTok{\textbackslash{}t}\StringTok{\textquotesingle{}}\NormalTok{, }\AttributeTok{row.names=}\ConstantTok{FALSE}\NormalTok{)}

\NormalTok{sig\_test\_pearson}\OtherTok{\textless{}{-}}\FunctionTok{lapply}\NormalTok{(yvar,}
       \ControlFlowTok{function}\NormalTok{(yvar, xvar, res\_diss)}
\NormalTok{       \{}
         \FunctionTok{cor.test}\NormalTok{(res\_diss[[xvar]], res\_diss[[yvar]], }\AttributeTok{method=}\StringTok{\textquotesingle{}pearson\textquotesingle{}}\NormalTok{) }\SpecialCharTok{\%\textgreater{}\%}
           \FunctionTok{tidy}\NormalTok{()}
\NormalTok{       \},}
\NormalTok{       xvar,}
\NormalTok{       res\_diss) }\SpecialCharTok{\%\textgreater{}\%}
  \FunctionTok{bind\_rows}\NormalTok{()}\SpecialCharTok{\%\textgreater{}\%}\FunctionTok{mutate}\NormalTok{(}\StringTok{\textquotesingle{}dissimiliarity\_index\textquotesingle{}}\OtherTok{=}\NormalTok{dissim\_names)}

\NormalTok{sig\_test\_pearson}
\end{Highlighting}
\end{Shaded}

\begin{verbatim}
## # A tibble: 15 x 9
##    estimate statistic  p.value parameter conf.low conf.high method   alternative
##       <dbl>     <dbl>    <dbl>     <int>    <dbl>     <dbl> <chr>    <chr>      
##  1   0.258      13.7  2.18e-41      2638  0.222      0.293  Pearson~ two.sided  
##  2   0.0360      1.85 6.45e- 2      2638 -0.00217    0.0740 Pearson~ two.sided  
##  3   0.175       9.14 1.20e-19      2638  0.138      0.212  Pearson~ two.sided  
##  4   0.273      14.6  1.94e-46      2638  0.238      0.308  Pearson~ two.sided  
##  5   0.316      17.1  2.95e-62      2638  0.281      0.350  Pearson~ two.sided  
##  6  -0.110      -5.69 1.38e- 8      2638 -0.148     -0.0723 Pearson~ two.sided  
##  7  -0.138      -7.16 1.02e-12      2638 -0.175     -0.100  Pearson~ two.sided  
##  8  -0.164      -8.55 1.97e-17      2638 -0.201     -0.127  Pearson~ two.sided  
##  9  -0.0825     -4.25 2.20e- 5      2638 -0.120     -0.0445 Pearson~ two.sided  
## 10   0.0356      1.83 6.74e- 2      2638 -0.00256    0.0736 Pearson~ two.sided  
## 11   0.0275      1.41 1.57e- 1      2638 -0.0106     0.0656 Pearson~ two.sided  
## 12   0.0739      3.81 1.44e- 4      2638  0.0359     0.112  Pearson~ two.sided  
## 13  -0.0241     -1.24 2.15e- 1      2638 -0.0622     0.0140 Pearson~ two.sided  
## 14   0.0738      3.80 1.48e- 4      2638  0.0357     0.112  Pearson~ two.sided  
## 15   0.112       5.79 7.65e- 9      2638  0.0743     0.150  Pearson~ two.sided  
## # ... with 1 more variable: dissimiliarity_index <chr>
\end{verbatim}

\begin{Shaded}
\begin{Highlighting}[]
\FunctionTok{write.table}\NormalTok{(sig\_test\_pearson, }\StringTok{"clipboard"}\NormalTok{, }\AttributeTok{sep=}\StringTok{\textquotesingle{}}\SpecialCharTok{\textbackslash{}t}\StringTok{\textquotesingle{}}\NormalTok{, }\AttributeTok{row.names=}\ConstantTok{FALSE}\NormalTok{)}
\end{Highlighting}
\end{Shaded}

What is the correlation in Seattle only between the dissimiliarity
indices and the percent single family zoning

\begin{Shaded}
\begin{Highlighting}[]
\CommentTok{\# db.connect \textless{}{-} function() \{}
\CommentTok{\#   elmer\_connection \textless{}{-} dbConnect(odbc(),}
\CommentTok{\#                                 driver = "SQL Server",}
\CommentTok{\#                                 server = "AWS{-}PROD{-}SQL\textbackslash{}\textbackslash{}Sockeye",}
\CommentTok{\#                                 database = "Elmer",}
\CommentTok{\#                                 trusted\_connection = "yes"}
\CommentTok{\#   )}
\CommentTok{\# \}}
\CommentTok{\# }
\CommentTok{\# \# a function to read tables and queries from Elmer}
\CommentTok{\# read.dt \textless{}{-} function(astring, type =c(\textquotesingle{}table\_name\textquotesingle{}, \textquotesingle{}sqlquery\textquotesingle{})) \{}
\CommentTok{\#   elmer\_connection \textless{}{-} db.connect()}
\CommentTok{\#   if (type == \textquotesingle{}table\_name\textquotesingle{}) \{}
\CommentTok{\#     dtelm \textless{}{-} dbReadTable(elmer\_connection, SQL(astring))}
\CommentTok{\#   \} else \{}
\CommentTok{\#     dtelm \textless{}{-} dbGetQuery(elmer\_connection, SQL(astring))}
\CommentTok{\#   \}}
\CommentTok{\#   dbDisconnect(elmer\_connection)}
\CommentTok{\#   dtelm}
\CommentTok{\# \}}
\CommentTok{\# }
\CommentTok{\# block\_group\_city\textless{}{-}read.dt("select pd.block\_group\_geoid10 from small\_areas.parcel\_dim pd where pd.city\_name\_2020 = \textquotesingle{}Seattle\textquotesingle{} group by pd.block\_group\_geoid10 order by pd.block\_group\_geoid10", \textquotesingle{}sqlquery\textquotesingle{})}
\CommentTok{\# }
\CommentTok{\# block\_group\_city \textless{}{-} block\_group\_city \%\textgreater{}\% mutate(block\_group\_geoid10 =as.character(block\_group\_geoid10))}
\CommentTok{\# res\_diss\textless{}{-}res\_diss \%\textgreater{}\% mutate(geoid10=as.character(geoid10))}
\CommentTok{\# res\_dissim\_seattle\textless{}{-} dplyr::left\_join(block\_group\_city ,res\_diss, by = c("block\_group\_geoid10"="geoid10"))}
\end{Highlighting}
\end{Shaded}

\begin{Shaded}
\begin{Highlighting}[]
\CommentTok{\# sig\_test\textless{}{-}lapply(yvar,}
\CommentTok{\#        function(yvar, xvar, res\_diss)}
\CommentTok{\#        \{}
\CommentTok{\#          print(yvar)}
\CommentTok{\#          cor.test(res\_dissim\_seattle[[xvar]], res\_dissim\_seattle[[yvar]], method=\textquotesingle{}spearman\textquotesingle{}) \%\textgreater{}\%}
\CommentTok{\#            tidy()}
\CommentTok{\#        \},}
\CommentTok{\#        xvar,}
\CommentTok{\#        res\_dissim\_seattle) \%\textgreater{}\%}
\CommentTok{\#   bind\_rows() \%\textgreater{}\%mutate(\textquotesingle{}dissimiliarity\_index\textquotesingle{}=dissim\_names)}
\CommentTok{\# }
\CommentTok{\# write.table(sig\_test, "clipboard", sep=\textquotesingle{}\textbackslash{}t\textquotesingle{}, row.names=FALSE)}
\CommentTok{\# }
\CommentTok{\# sig\_test\_pearson\textless{}{-}lapply(yvar,}
\CommentTok{\#        function(yvar, xvar, res\_diss)}
\CommentTok{\#        \{}
\CommentTok{\#          cor.test(res\_dissim\_seattle[[xvar]], res\_diss\_seattle[[yvar]], method=\textquotesingle{}pearson\textquotesingle{}) \%\textgreater{}\%}
\CommentTok{\#            tidy()}
\CommentTok{\#        \},}
\CommentTok{\#        xvar,}
\CommentTok{\#        res\_diss\_seattle) \%\textgreater{}\%}
\CommentTok{\#   bind\_rows()\%\textgreater{}\%mutate(\textquotesingle{}dissimiliarity\_index\textquotesingle{}=dissim\_names)}
\CommentTok{\# }
\CommentTok{\# sig\_test\_pearson}
\CommentTok{\# write.table(sig\_test\_pearson, "clipboard", sep=\textquotesingle{}\textbackslash{}t\textquotesingle{}, row.names=FALSE)}
\end{Highlighting}
\end{Shaded}

This is a summary of the the range of the percent single family
variable. There are many block groups with 100\% single family zoning,
and many with 0\% single family zoning. Higher shares of single family
zoning are more common, such as greater than 75\%, than lower shares
such as 25\%

\begin{Shaded}
\begin{Highlighting}[]
\FunctionTok{ggplot}\NormalTok{(res\_diss, }\FunctionTok{aes}\NormalTok{(}\AttributeTok{x=}\NormalTok{prct\_sf)) }\SpecialCharTok{+}\FunctionTok{geom\_histogram}\NormalTok{()}
\end{Highlighting}
\end{Shaded}

\begin{verbatim}
## `stat_bin()` using `bins = 30`. Pick better value with `binwidth`.
\end{verbatim}

\includegraphics{dissim_zoning_analysis_9_21_files/figure-latex/unnamed-chunk-4-1.pdf}

In this summary, I am plotting histograms for the values taken on by
each dissimilarity index. The distributions of each dissimilarity index
can tell a story. I think we need to work together to figure out what it
means.

\begin{Shaded}
\begin{Highlighting}[]
\ControlFlowTok{for}\NormalTok{(name }\ControlFlowTok{in}\NormalTok{ dissim\_names)\{}
\FunctionTok{print}\NormalTok{(}\FunctionTok{ggplot}\NormalTok{(res\_diss, }\FunctionTok{aes\_string}\NormalTok{(}\AttributeTok{x=}\NormalTok{name)) }\SpecialCharTok{+} \FunctionTok{geom\_histogram}\NormalTok{())}
\NormalTok{\}}
\end{Highlighting}
\end{Shaded}

\begin{verbatim}
## `stat_bin()` using `bins = 30`. Pick better value with `binwidth`.
\end{verbatim}

\includegraphics{dissim_zoning_analysis_9_21_files/figure-latex/unnamed-chunk-5-1.pdf}

\begin{verbatim}
## `stat_bin()` using `bins = 30`. Pick better value with `binwidth`.
\end{verbatim}

\includegraphics{dissim_zoning_analysis_9_21_files/figure-latex/unnamed-chunk-5-2.pdf}

\begin{verbatim}
## `stat_bin()` using `bins = 30`. Pick better value with `binwidth`.
\end{verbatim}

\includegraphics{dissim_zoning_analysis_9_21_files/figure-latex/unnamed-chunk-5-3.pdf}

\begin{verbatim}
## `stat_bin()` using `bins = 30`. Pick better value with `binwidth`.
\end{verbatim}

\includegraphics{dissim_zoning_analysis_9_21_files/figure-latex/unnamed-chunk-5-4.pdf}

\begin{verbatim}
## `stat_bin()` using `bins = 30`. Pick better value with `binwidth`.
\end{verbatim}

\includegraphics{dissim_zoning_analysis_9_21_files/figure-latex/unnamed-chunk-5-5.pdf}

\begin{verbatim}
## `stat_bin()` using `bins = 30`. Pick better value with `binwidth`.
\end{verbatim}

\includegraphics{dissim_zoning_analysis_9_21_files/figure-latex/unnamed-chunk-5-6.pdf}

\begin{verbatim}
## `stat_bin()` using `bins = 30`. Pick better value with `binwidth`.
\end{verbatim}

\includegraphics{dissim_zoning_analysis_9_21_files/figure-latex/unnamed-chunk-5-7.pdf}

\begin{verbatim}
## `stat_bin()` using `bins = 30`. Pick better value with `binwidth`.
\end{verbatim}

\includegraphics{dissim_zoning_analysis_9_21_files/figure-latex/unnamed-chunk-5-8.pdf}

\begin{verbatim}
## `stat_bin()` using `bins = 30`. Pick better value with `binwidth`.
\end{verbatim}

\includegraphics{dissim_zoning_analysis_9_21_files/figure-latex/unnamed-chunk-5-9.pdf}

\begin{verbatim}
## `stat_bin()` using `bins = 30`. Pick better value with `binwidth`.
\end{verbatim}

\includegraphics{dissim_zoning_analysis_9_21_files/figure-latex/unnamed-chunk-5-10.pdf}

\begin{verbatim}
## `stat_bin()` using `bins = 30`. Pick better value with `binwidth`.
\end{verbatim}

\includegraphics{dissim_zoning_analysis_9_21_files/figure-latex/unnamed-chunk-5-11.pdf}

\begin{verbatim}
## `stat_bin()` using `bins = 30`. Pick better value with `binwidth`.
\end{verbatim}

\includegraphics{dissim_zoning_analysis_9_21_files/figure-latex/unnamed-chunk-5-12.pdf}

\begin{verbatim}
## `stat_bin()` using `bins = 30`. Pick better value with `binwidth`.
\end{verbatim}

\includegraphics{dissim_zoning_analysis_9_21_files/figure-latex/unnamed-chunk-5-13.pdf}

\begin{verbatim}
## `stat_bin()` using `bins = 30`. Pick better value with `binwidth`.
\end{verbatim}

\includegraphics{dissim_zoning_analysis_9_21_files/figure-latex/unnamed-chunk-5-14.pdf}

\begin{verbatim}
## `stat_bin()` using `bins = 30`. Pick better value with `binwidth`.
\end{verbatim}

\includegraphics{dissim_zoning_analysis_9_21_files/figure-latex/unnamed-chunk-5-15.pdf}

Now I am plotting the relationship between the prct\_sf and the
dissimilarity indices. It looks pretty noisy as the correlation
coefficients imply.

\begin{Shaded}
\begin{Highlighting}[]
\ControlFlowTok{for}\NormalTok{(name }\ControlFlowTok{in}\NormalTok{ dissim\_names)\{}
\FunctionTok{print}\NormalTok{(}\FunctionTok{ggplot}\NormalTok{(res\_diss, }\FunctionTok{aes\_string}\NormalTok{(}\AttributeTok{x=}\StringTok{\textquotesingle{}prct\_sf\textquotesingle{}}\NormalTok{, }\AttributeTok{y=}\NormalTok{name))}\SpecialCharTok{+}
  \FunctionTok{geom\_point}\NormalTok{()}\SpecialCharTok{+}
  \FunctionTok{geom\_smooth}\NormalTok{(}\AttributeTok{method=}\NormalTok{lm, }\AttributeTok{formula=}\StringTok{\textquotesingle{}y \textasciitilde{}poly(x,3)\textquotesingle{}}\NormalTok{))}
\NormalTok{\}}
\end{Highlighting}
\end{Shaded}

\includegraphics{dissim_zoning_analysis_9_21_files/figure-latex/unnamed-chunk-6-1.pdf}
\includegraphics{dissim_zoning_analysis_9_21_files/figure-latex/unnamed-chunk-6-2.pdf}
\includegraphics{dissim_zoning_analysis_9_21_files/figure-latex/unnamed-chunk-6-3.pdf}
\includegraphics{dissim_zoning_analysis_9_21_files/figure-latex/unnamed-chunk-6-4.pdf}
\includegraphics{dissim_zoning_analysis_9_21_files/figure-latex/unnamed-chunk-6-5.pdf}
\includegraphics{dissim_zoning_analysis_9_21_files/figure-latex/unnamed-chunk-6-6.pdf}
\includegraphics{dissim_zoning_analysis_9_21_files/figure-latex/unnamed-chunk-6-7.pdf}
\includegraphics{dissim_zoning_analysis_9_21_files/figure-latex/unnamed-chunk-6-8.pdf}
\includegraphics{dissim_zoning_analysis_9_21_files/figure-latex/unnamed-chunk-6-9.pdf}
\includegraphics{dissim_zoning_analysis_9_21_files/figure-latex/unnamed-chunk-6-10.pdf}
\includegraphics{dissim_zoning_analysis_9_21_files/figure-latex/unnamed-chunk-6-11.pdf}
\includegraphics{dissim_zoning_analysis_9_21_files/figure-latex/unnamed-chunk-6-12.pdf}
\includegraphics{dissim_zoning_analysis_9_21_files/figure-latex/unnamed-chunk-6-13.pdf}
\includegraphics{dissim_zoning_analysis_9_21_files/figure-latex/unnamed-chunk-6-14.pdf}
\includegraphics{dissim_zoning_analysis_9_21_files/figure-latex/unnamed-chunk-6-15.pdf}

\end{document}
