\PassOptionsToPackage{unicode=true}{hyperref} % options for packages loaded elsewhere
\PassOptionsToPackage{hyphens}{url}
%
\documentclass[
]{article}
\usepackage{lmodern}
\usepackage{amssymb,amsmath}
\usepackage{ifxetex,ifluatex}
\ifnum 0\ifxetex 1\fi\ifluatex 1\fi=0 % if pdftex
  \usepackage[T1]{fontenc}
  \usepackage[utf8]{inputenc}
  \usepackage{textcomp} % provides euro and other symbols
\else % if luatex or xelatex
  \usepackage{unicode-math}
  \defaultfontfeatures{Scale=MatchLowercase}
  \defaultfontfeatures[\rmfamily]{Ligatures=TeX,Scale=1}
\fi
% use upquote if available, for straight quotes in verbatim environments
\IfFileExists{upquote.sty}{\usepackage{upquote}}{}
\IfFileExists{microtype.sty}{% use microtype if available
  \usepackage[]{microtype}
  \UseMicrotypeSet[protrusion]{basicmath} % disable protrusion for tt fonts
}{}
\makeatletter
\@ifundefined{KOMAClassName}{% if non-KOMA class
  \IfFileExists{parskip.sty}{%
    \usepackage{parskip}
  }{% else
    \setlength{\parindent}{0pt}
    \setlength{\parskip}{6pt plus 2pt minus 1pt}}
}{% if KOMA class
  \KOMAoptions{parskip=half}}
\makeatother
\usepackage{xcolor}
\IfFileExists{xurl.sty}{\usepackage{xurl}}{} % add URL line breaks if available
\IfFileExists{bookmark.sty}{\usepackage{bookmark}}{\usepackage{hyperref}}
\hypersetup{
  pdftitle={Calculating trip rates},
  pdfauthor={Polina Butrina},
  pdfborder={0 0 0},
  breaklinks=true}
\urlstyle{same}  % don't use monospace font for urls
\usepackage[margin=1in]{geometry}
\usepackage{color}
\usepackage{fancyvrb}
\newcommand{\VerbBar}{|}
\newcommand{\VERB}{\Verb[commandchars=\\\{\}]}
\DefineVerbatimEnvironment{Highlighting}{Verbatim}{commandchars=\\\{\}}
% Add ',fontsize=\small' for more characters per line
\usepackage{framed}
\definecolor{shadecolor}{RGB}{248,248,248}
\newenvironment{Shaded}{\begin{snugshade}}{\end{snugshade}}
\newcommand{\AlertTok}[1]{\textcolor[rgb]{0.94,0.16,0.16}{#1}}
\newcommand{\AnnotationTok}[1]{\textcolor[rgb]{0.56,0.35,0.01}{\textbf{\textit{#1}}}}
\newcommand{\AttributeTok}[1]{\textcolor[rgb]{0.77,0.63,0.00}{#1}}
\newcommand{\BaseNTok}[1]{\textcolor[rgb]{0.00,0.00,0.81}{#1}}
\newcommand{\BuiltInTok}[1]{#1}
\newcommand{\CharTok}[1]{\textcolor[rgb]{0.31,0.60,0.02}{#1}}
\newcommand{\CommentTok}[1]{\textcolor[rgb]{0.56,0.35,0.01}{\textit{#1}}}
\newcommand{\CommentVarTok}[1]{\textcolor[rgb]{0.56,0.35,0.01}{\textbf{\textit{#1}}}}
\newcommand{\ConstantTok}[1]{\textcolor[rgb]{0.00,0.00,0.00}{#1}}
\newcommand{\ControlFlowTok}[1]{\textcolor[rgb]{0.13,0.29,0.53}{\textbf{#1}}}
\newcommand{\DataTypeTok}[1]{\textcolor[rgb]{0.13,0.29,0.53}{#1}}
\newcommand{\DecValTok}[1]{\textcolor[rgb]{0.00,0.00,0.81}{#1}}
\newcommand{\DocumentationTok}[1]{\textcolor[rgb]{0.56,0.35,0.01}{\textbf{\textit{#1}}}}
\newcommand{\ErrorTok}[1]{\textcolor[rgb]{0.64,0.00,0.00}{\textbf{#1}}}
\newcommand{\ExtensionTok}[1]{#1}
\newcommand{\FloatTok}[1]{\textcolor[rgb]{0.00,0.00,0.81}{#1}}
\newcommand{\FunctionTok}[1]{\textcolor[rgb]{0.00,0.00,0.00}{#1}}
\newcommand{\ImportTok}[1]{#1}
\newcommand{\InformationTok}[1]{\textcolor[rgb]{0.56,0.35,0.01}{\textbf{\textit{#1}}}}
\newcommand{\KeywordTok}[1]{\textcolor[rgb]{0.13,0.29,0.53}{\textbf{#1}}}
\newcommand{\NormalTok}[1]{#1}
\newcommand{\OperatorTok}[1]{\textcolor[rgb]{0.81,0.36,0.00}{\textbf{#1}}}
\newcommand{\OtherTok}[1]{\textcolor[rgb]{0.56,0.35,0.01}{#1}}
\newcommand{\PreprocessorTok}[1]{\textcolor[rgb]{0.56,0.35,0.01}{\textit{#1}}}
\newcommand{\RegionMarkerTok}[1]{#1}
\newcommand{\SpecialCharTok}[1]{\textcolor[rgb]{0.00,0.00,0.00}{#1}}
\newcommand{\SpecialStringTok}[1]{\textcolor[rgb]{0.31,0.60,0.02}{#1}}
\newcommand{\StringTok}[1]{\textcolor[rgb]{0.31,0.60,0.02}{#1}}
\newcommand{\VariableTok}[1]{\textcolor[rgb]{0.00,0.00,0.00}{#1}}
\newcommand{\VerbatimStringTok}[1]{\textcolor[rgb]{0.31,0.60,0.02}{#1}}
\newcommand{\WarningTok}[1]{\textcolor[rgb]{0.56,0.35,0.01}{\textbf{\textit{#1}}}}
\usepackage{graphicx,grffile}
\makeatletter
\def\maxwidth{\ifdim\Gin@nat@width>\linewidth\linewidth\else\Gin@nat@width\fi}
\def\maxheight{\ifdim\Gin@nat@height>\textheight\textheight\else\Gin@nat@height\fi}
\makeatother
% Scale images if necessary, so that they will not overflow the page
% margins by default, and it is still possible to overwrite the defaults
% using explicit options in \includegraphics[width, height, ...]{}
\setkeys{Gin}{width=\maxwidth,height=\maxheight,keepaspectratio}
\setlength{\emergencystretch}{3em}  % prevent overfull lines
\providecommand{\tightlist}{%
  \setlength{\itemsep}{0pt}\setlength{\parskip}{0pt}}
\setcounter{secnumdepth}{-2}
% Redefines (sub)paragraphs to behave more like sections
\ifx\paragraph\undefined\else
  \let\oldparagraph\paragraph
  \renewcommand{\paragraph}[1]{\oldparagraph{#1}\mbox{}}
\fi
\ifx\subparagraph\undefined\else
  \let\oldsubparagraph\subparagraph
  \renewcommand{\subparagraph}[1]{\oldsubparagraph{#1}\mbox{}}
\fi

% set default figure placement to htbp
\makeatletter
\def\fps@figure{htbp}
\makeatother


\title{Calculating trip rates}
\author{Polina Butrina}
\date{10/11/2020}

\begin{document}
\maketitle

\hypertarget{r-markdown}{%
\subsection{R Markdown}\label{r-markdown}}

For general HHTS data analysis tips and tricks, please refer to this
intro from RSG.

There are several analyses you can do with trips: • Find trip rates •
Find a number of trips

\hypertarget{find-trip-rates}{%
\subsubsection{Find Trip Rates}\label{find-trip-rates}}

Trips rates (or the number of trips per day among groups) can be found
using following steps:

\begin{enumerate}
\def\labelenumi{\arabic{enumi}.}
\tightlist
\item
  Filter trip table to just shopping trips
\end{enumerate}

\begin{Shaded}
\begin{Highlighting}[]
\NormalTok{trip.query =}\StringTok{ }\KeywordTok{paste}\NormalTok{(}\StringTok{"SELECT * FROM HHSurvey.v_trips_2017_2019"}\NormalTok{)}
\NormalTok{trips =}\StringTok{ }\KeywordTok{read.dt}\NormalTok{(trip.query, }\StringTok{'sqlquery'}\NormalTok{)}

\NormalTok{shop_trips <-}\StringTok{ }\NormalTok{trips }\OperatorTok\StringTok{ }\KeywordTok{filter}\NormalTok{(d_purp_cat}\OperatorTok{==}\StringTok{'Shop'}\NormalTok{)}
\end{Highlighting}
\end{Shaded}

\begin{enumerate}
\def\labelenumi{\arabic{enumi}.}
\setcounter{enumi}{1}
\tightlist
\item
  Sum trip\_wt\_combined multiplied
\end{enumerate}

\begin{Shaded}
\begin{Highlighting}[]
\KeywordTok{sum}\NormalTok{(shop_trips}\OperatorTok{$}\NormalTok{trip_wt_combined)}
\end{Highlighting}
\end{Shaded}

\begin{verbatim}
## [1] 2283929
\end{verbatim}

\begin{Shaded}
\begin{Highlighting}[]
\NormalTok{shop_trips_gender<-}\KeywordTok{create_table_one_var}\NormalTok{(}\StringTok{'gender'}\NormalTok{, shop_trips, }\StringTok{'trip'}\NormalTok{)}
\end{Highlighting}
\end{Shaded}

\begin{verbatim}
## `summarise()` ungrouping output (override with `.groups` argument)
\end{verbatim}

\begin{Shaded}
\begin{Highlighting}[]
\NormalTok{shop_trips_gender}
\end{Highlighting}
\end{Shaded}

\begin{verbatim}
## # A tibble: 3 x 10
##   gender     n sum_wt_comb sum_wt_2017 sum_wt_2019 perc_comb perc_2017
##   <chr>  <int>       <dbl>       <dbl>       <dbl>     <dbl>     <dbl>
## 1 Female  7053    1241815.     891681.    1072194.   55.7      60.1   
## 2 Male    5752     984365.     592035.    1120854.   44.2      39.9   
## 3 Anoth~    62       1675.        195.       4652.    0.0752    0.0132
## # ... with 3 more variables: perc_2019 <dbl>, delta <dbl>, MOE <dbl>
\end{verbatim}

\begin{enumerate}
\def\labelenumi{\arabic{enumi}.}
\setcounter{enumi}{2}
\tightlist
\item
  Merge trip counts with the day table
\end{enumerate}

\begin{Shaded}
\begin{Highlighting}[]
\NormalTok{days<-}\KeywordTok{paste}\NormalTok{(}\StringTok{"SELECT * FROM HHSurvey.v_days_2017_2019_in_house"}\NormalTok{)}
\NormalTok{person_days=}\KeywordTok{read.dt}\NormalTok{(days, }\StringTok{'sqlquery'}\NormalTok{)}

\NormalTok{person.query =}\StringTok{ }\KeywordTok{paste}\NormalTok{(}\StringTok{"SELECT * FROM HHSurvey.v_persons_2017_2019"}\NormalTok{)}
\NormalTok{person =}\StringTok{ }\KeywordTok{read.dt}\NormalTok{(person.query, }\StringTok{'sqlquery'}\NormalTok{)}

\NormalTok{person_days_}\DecValTok{2}\NormalTok{<-}\KeywordTok{merge}\NormalTok{(person_days, person, }\DataTypeTok{by.x=}\StringTok{'personid'}\NormalTok{, }\DataTypeTok{by.y=}\StringTok{'person_dim_id'}\NormalTok{)}
\end{Highlighting}
\end{Shaded}

\begin{enumerate}
\def\labelenumi{\arabic{enumi}.}
\setcounter{enumi}{3}
\tightlist
\item
  Summarize day trips by gender
\end{enumerate}

\begin{Shaded}
\begin{Highlighting}[]
\NormalTok{person_day_gender <-}\StringTok{ }\NormalTok{person_days_}\DecValTok{2} \OperatorTok\StringTok{ }\KeywordTok{group_by}\NormalTok{(gender) }\OperatorTok
\StringTok{                      }\KeywordTok{summarise}\NormalTok{(}\DataTypeTok{n=}\KeywordTok{n}\NormalTok{(), }\DataTypeTok{day_combined =} \KeywordTok{sum}\NormalTok{(hh_day_wt_combined.x))}
\end{Highlighting}
\end{Shaded}

\begin{verbatim}
## `summarise()` ungrouping output (override with `.groups` argument)
\end{verbatim}

\begin{Shaded}
\begin{Highlighting}[]
\NormalTok{person_day_gender}
\end{Highlighting}
\end{Shaded}

\begin{verbatim}
## # A tibble: 4 x 3
##   gender                   n day_combined
##   <chr>                <int>        <dbl>
## 1 Another                113        5999.
## 2 Female               15393     1967804.
## 3 Male                 14920     1950827.
## 4 Prefer not to answer   456      126951.
\end{verbatim}

\begin{enumerate}
\def\labelenumi{\arabic{enumi}.}
\setcounter{enumi}{4}
\tightlist
\item
  Calculate trip rate as sum of shopping trips divided by the number of
  weighted person days
\end{enumerate}

\begin{Shaded}
\begin{Highlighting}[]
\NormalTok{shop_trips_}\DecValTok{3}\NormalTok{ <-}\StringTok{ }\KeywordTok{merge}\NormalTok{(shop_trips_gender, person_day_gender, }\DataTypeTok{by =} \StringTok{'gender'}\NormalTok{)}
\NormalTok{shop_trips_}\DecValTok{3} \OperatorTok\StringTok{ }\KeywordTok{mutate}\NormalTok{(}\DataTypeTok{trip_rate =}\NormalTok{ sum_wt_comb}\OperatorTok{/}\NormalTok{day_combined) }\OperatorTok\StringTok{ }\KeywordTok{select}\NormalTok{(gender,trip_rate)}
\end{Highlighting}
\end{Shaded}

\begin{verbatim}
##    gender trip_rate
## 1 Another 0.2792599
## 2  Female 0.6310663
## 3    Male 0.5045885
\end{verbatim}

Note that the \texttt{echo\ =\ FALSE} parameter was added to the code
chunk to prevent printing of the R code that generated the plot.

\end{document}
